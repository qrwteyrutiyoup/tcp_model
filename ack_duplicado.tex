%Equipe: Vinicius, Luis Ribeiro
%Livro: Seção 5.3.2, p.131

\subsection{ACKs Duplicados}
A modelagem do protocolo TCP é feita sobre o algoritmo de controle congestionamento TCP Reno.
Neste algoritmo, após o \textit{Slow Start}, o tamanho da janela de congestionamento ($W$) é 
incrementado em $1/W$ a cada \textit{ACK} recebido e decrementado a cada indicação de perda 
de pacote percebida. O valor da redução no tamanho da janela varia de acordo com o tipo de indicação de
perda de pacote: triplo \textit{ACK} duplicado ou \textit{timeout}. No primeiro caso o tamanho
da janela é dividido pela metade do seu valor corrente, no segundo a janela é reduzida a um 
pacote. Nesta seção consideraremos que perdas de pacotes são indicadas apenas por triplos 
\textit{ACK's} duplicados (TD).

O comportamento do protocolo TCP é modelado em termos de rodadas (\textit{rounds}). Uma rodada
se inicia com a transmissão de $W$ pacotes de uma janela. Uma vez que todos os pacotes tenham sido 
transmitidos, o remetente passa a aguardar a recepção de \textit{ACK's}. O recebimento do primeiro 
\textit{ACK} marca o fim da rodada atual e início da próxima. Note que a cada rodada o tamanho da 
janela é incrementado em um pacote.

A figura \ref{fig:tdp_macro} apresenta o comportamento de uma conexão TCP, descrito acima, em mais detalhes. 
Defini-se período TD (TDP) como o intervalo entre duas indicações de perda de pacote por triplo \textit{ACK} 
duplicado (TD). Durante este intervalo, o tamanho da janela é incrementado de uma unidade a cada rodada. 
No fim de cada TDP$_{i}$ o tamanho da janela é dividido ao meio.

\begin{figure}[ht]
  \centering
  \includegraphics[scale=0.8]{./figs/tdp_macro.pdf}
  \caption{Evolução do tamanho da janela quando perdas de pacotes são TDs.}
  \label{fig:tdp_macro}
\end{figure}

O objetivo é encontrar o taxa de transferência de dados no estado estacionário. Assim, para o $i$-ésimo 
período TD (TDP$_{i}$) define-se Y$_i$ como a quantidade de pacotes enviados no período, A$_i$ a duração 
do período e W$_i$ o tamanho da janela no final do período. Dessa forma, a taxa média de tranferência é
dada pela divisão entre o número médio (esperança) de pacotes transmitidos em um TDP pelo tempo médio 
(esperança) de duração de um intervalo TD, ver \eqref{eq:throughput}.

\begin{equation} \label{eq:throughput}
\overline{X}=\frac{E[Y]}{E[A]}
\end{equation}

Agora torna-se necessário deduzir expressões para $Y$ e $A$ a fim de deterninar $\overline{X}$. Assim, 
analisemos em mais detalhes um intervalo TDP$_i$, ver figura \ref{fig:tdp_detalhe}.
Um período TD se inicia logo após uma indicação de perda de pacote por triplo \textit{ACK} duplicado, 
onde o tamanho da janela corrente passa a ser a metade do tamanho da janela anterior $W_{i-1} / 2$. 
Depois disso, a janela é incrementada a cada rodada até que uma nova perda seja identificada por triplo 
\textit{ACK} duplicado. Denota-se por $\alpha_i$ o primeiro pacote perdido e por R$_i$ a rodada onde a 
perda acorreu. 

\begin{figure}[ht]
  \centering
  \includegraphics[scale=0.8]{./figs/tdp_detalhe.pdf}
  \caption{Pacotes enviados durante um período TD (TDP$_{i}$).}
  \label{fig:tdp_detalhe}
\end{figure}

Para calcularmos a quantidade de pacotes transmitidos em um TDP$_i$, é necessário descobrirmos quantos pacotes
são enviados após o $\alpha_i$. Assim, digamos que na janela W$_i$, $\alpha_i$ seja o $n$-ésimo pacote a ser
enviado (ver figura \ref{fig:tdp_detalhe}). Logo, ainda na mesma rodada  são transmitidos $W_i - n$ pacotes 
imediatamente após $\alpha_i$. Como antes de $\alpha_i$ foram corretamente enviados e reconhecidos $n - 1$ pacotes 
na rodada R$_i$, na última rodada ainda são transmitidos $n - 1$ pacotes. Assim o número total de pacotes enviados
após $\alpha_i$ é: $W_i - n + n -1 = W_i - 1$. Daí, temos que o número total de pacotes transmitidos em um
TDP$_i$ é dado por: $Y_i = \alpha_i + W_i - 1$. 

Considerando que $\alpha$ e $W$ são independentes entre si,
pode-se utilizar a esperança obtendo \eqref{eq:EY1}.

\begin{equation} \label{eq:EY1}
E[Y]=E[\alpha] + E[W] - 1
\end{equation}

Precisa-se calcular $E[\alpha]$. Considerando que a perda de um pacote em um round é independente da perda de 
pacotes em quaisquer outros rounds, \{$\alpha_i$\}$_i$ é uma variável aleatória identicamente distribuída. 
Isto é, o número de pacotes $\alpha_i$ transmitidos em um TDP$_i$ é independente da quantidade de pacotes
transmitidos em outros períodos. Assim, a probabilidade de $\alpha_i = k$ é igual a probabilidade de que
exatamente $k - 1$ pacotes sejam transmitidos corretamente até que um pacote seja perdido, ver 
\eqref{eq:P_alpha}.

\begin{equation} \label{eq:P_alpha}
P[\alpha=k]=(1 - p)^{k - 1} * p
\end{equation}

A esperança ou média de $\alpha$ é dada por \eqref{eq:E_alpha}.

\begin{align} \label{eq:E_alpha}
\nonumber E[\alpha] &= \sum_{k=1}^\infty (1 - p)^{k - 1} * p * k \\
\nonumber &= \sum_{k=1}^\infty \frac{(1 - p)^{k}}{1 - p} * p * k\\ 
\nonumber &= \frac{p}{1 - p} * \sum_{k=1}^\infty (1 - p)^k * k \\ 
\nonumber &= \frac{p}{1 - p} * \frac{1 - p}{(1 - (1 - p))^2} \\
&= \frac{1}{p}
\end{align}

Substituindo \eqref{eq:E_alpha} em \eqref{eq:EY1}, temos \eqref{eq:EY2}. 

\begin{align} \label{eq:EY2}
\nonumber E[Y] &= E[\alpha] + E[W] - 1 \\
&= \frac{1 - p}{p} + E[W]
\end{align}

Pode-se obter uma outra expressão para o número de pacotes transmitidos ($Y$) analisando
a evolução do tamanho da janela de transmissão ($W$) durante as rodadas de um TDP$_i$.
Na primeira rodada de um novo período TD o tamanho da janela de transmissão é a metade
do tamanho da janela anterior (ver figura \ref{fig:tdp_detalhe}). Depois o tamanho da janela
é incrementado a cada rodada até a perda do primeiro pacote em R{$_i$}. Assim, o tamanho 
final da janela (W$_i$) é dado por \eqref{eq:Wi}.

\begin{equation} \label{eq:Wi}
Wi=\frac{W_{i-1}}{2} + R_i
\end{equation}

Considerando que \{W$_i$\} e \{R$_i$\} são variáveis independentes, pode-se utilizar a 
esperança em ambos os lados de \eqref{eq:Wi} obtendo \eqref{eq:EW1}.

\begin{align} \label{eq:EW1}
\nonumber & E[W]=\frac{E[W]}{2} + E[R] \\
& E[W]= 2 * E[R]
\end{align}

A quantidade de pacotes transmitidos (Y$_i$) durante um TDP$_i$ pode ser calculada somando-se 
a número de pacotes enviados em cada uma de suas rodadas. O número de pacotes enviados por 
rodada é indicado pelo tamanho da janela de transmissão que é icrementada de uma unidade a 
cada rodada até a perda do primeiro pacote na rodada R$_i$. Na rodada seguinte, última rodada 
em TDP$_i$, apenas $\beta_i$ pacotes são transmitidos em resposta aos pacotes corretamente 
enviados antes da perda no round R$_i$. Assim, o somatório dos tamanhos das janelas em um 
TDP$_i$ indica a quantidade de pacotes transmitidos, ver \eqref{eq:Y2}.

\begin{align} \label{eq:Y2}
\nonumber Y_i &= \sum_{k = 0}^{R_i - 1} (\frac{W_{i - 1}}{2} + k) + \beta_i \\
\nonumber &= \sum_{k = 0}^{R_i - 1} \frac{W_{i - 1}}{2} + \sum_{k = 0}^{R_i - 1} k + \beta_i \\
\nonumber &= R_i * \frac{W_{i - 1}}{2} + \frac{R_i - 1}{2} * R_i + \beta_i \\
\nonumber &= \frac{R_i}{2} * (W_{i - 1} + R_i - 1) + \beta_i \text{ , de \eqref{eq:Wi} temos:$R_i = W_i - W_{i-1}/2$} \\
\nonumber &= \frac{R_i}{2} * (W_{i - 1} + W_i - \frac{W_{i-1}}{2} - 1) + \beta_i \\
&= \frac{R_i}{2} * (\frac{W_{i-1}}{2} + W_i - 1) + \beta_i 
\end{align}

Considerando que o número de rodadas é independente do tamanho da janela, pode-se utilizar
a esperança em cada variável de \eqref{eq:Y2}, obtendo \eqref{eq:EY3}.

\begin{equation} \label{eq:EY3}
E[Y] = \frac{E[R]}{2} * (\frac{E[W]}{2} + E[W] - 1) + E[\beta] 
\end{equation}

Considerando que o número de pacotes enviados na última rodada é uniformemente distribuído entre
1 e $W_i - 1$, temos $E[\beta] = E[W]/2$. Esta consideração é válida porque todos os pacotes tem 
a mesma probabilidade $p$ de serem perdidos. Além disso, de \eqref{eq:EW1} temos que: $E[R]=E[W]/2$.
Substituindo estes resultados em \eqref{eq:EY3}, obtemos \eqref{eq:EY4}

\begin{equation} \label{eq:EY4}
E[Y] = \frac{E[W]}{4} * (\frac{E[W]}{2} + E[W] - 1) + \frac{E[W]}{2}
\end{equation}

Igualando \eqref{eq:EY2} e \eqref{eq:EY4}, temos \eqref{eq:EW2}:

\begin{align} \label{eq:EW2}
\nonumber & \frac{1 - p}{p} + E[W] = \frac{E[W]}{4} * (\frac{E[W]}{2} + E[W] - 1) + \frac{E[W]}{2} \\
\nonumber & \frac{1 - p}{p} + E[W] = \frac{E[W]^2}{8} + \frac{E[W]^2}{4} - \frac{E[W]}{4} + \frac{E[W]}{2} \\
\nonumber & \frac{E[W]^2}{8} + \frac{E[W]^2}{4} - \frac{E[W]}{4} + \frac{E[W]}{2} - E[W] + \frac{p - 1}{p} = 0 \text{   $(*8)$}\\
& 3E[W]^2 - 6E[W] - \frac{8(p - 1)}{p} = 0
\end{align}

Calculando a raiz da equação \eqref{eq:EW2} e selecionando a positiva, temos \eqref{eq:EW3}:
\begin{equation} \label{eq:EW3}
E[W] = 1 + \sqrt{\frac{8}{3p} - \frac{5}{3}}
\end{equation}

